\documentclass{article}
\usepackage{fancyhdr}
\usepackage{tabu}
\usepackage[bottom=1in]{geometry}
\usepackage{multirow}
\usepackage{graphicx}
\usepackage{parskip}
\usepackage{pdfpages}
\setlength{\parindent}{15pt}

%\pagestyle{fancy}

\begin{document}
%USAFA DFEC Header
	\noindent \begin{tabu} to \textwidth{l X[c] r}
	\multirow{5}{*}{\includegraphics[width=0.75in]{DOD.jpg}} & 
	\textbf{United States Air Force Academy} &  
	\multirow{5}{*}{\includegraphics[width=0.75in]{USAFA.png}}\\
	& \textbf{Department of Electrical and Computer Engineering} & \\
	& \tiny{USAF ACADEMY, COLORADO}\\
	\\ \\ \\
	\end{tabu}

%Start text here
	\hfill 1 Oct 2013
	\centerline{\LARGE{\textbf{Why Oxford?}}} \hspace{0pt} \\
	\centerline{\Large{Captain Ryan Silva}}
	\centerline{\large{Assistant Professor}}
	\centerline{\large{Department of Electrical and Computer Engineering}}
	\centerline{\large{United States Air Force Academy}} \hspace{0pt} \\
\indent My ultimate goal in pursuing the Oxford experience is be to bring world-class
Biomedical Engineering research opportunities to USAFA. The core academic
strengths of Oxford's Center for Affordable Healthcare Technology (OxCAHT) and
the open-source nature of their research paired with Oxford's matchless
geopolitical landscape make the University of Oxford the only English-speaking
university in the world with whom my research aspirations can be accomplished.
As a project mentor for the FalconWorks-funded senior capstone project
NeuMimic, I bring valuable experience leading a successful multi-disciplinary,
systems-level biomedical engineering research project. I would come armed with 
valuable leadership experience paired with technical expertise in signal 
processing and embedded systems, thereby poised to become an immediate contributor;
it is important to note that the director of OxCAHT, Dr. Gari Clifford, agrees
with my assessment. Dr. Clifford, after a \emph{very} thorough Skype interview,
offered to review a research proposal for appropriateness and feasibility of
completion in 3 years. The signed MFR can be found in Attachment 1
on page~\pageref{sec:prop}. It seems as though this opportunity was tailored to 
suit my strengths and interests! 
 
There are few fields that offer more inspiring opportunities for systems-level 
research as biomedical engineering, especially the field of diagnostics.
Diagnostics is a very challenging discipline as it usually involves
prohibitively expensive and immobile equipment that regularly forces patients
to travel great distances at great cost in order to access care. These traits
currently inherent to medical diagnostics regularly exclude certain
populations from receiving proper
treatment. Affected populations can range from military members operating in an
expeditionary environment to rural communities, especially in the developing
world. My proposed research seeks to develop solutions to the critical need for
point of care devices that are cheap, power-efficient, reliable and
transportable. The immediate impact of this research is clear: it will allow
individuals to access healthcare that otherwise would not be able. Oxford's
centuries-old geopolitical affiliations allow this type of research to progress 
at a pace unrivaled at any other university or center. This is proven by the
expectations of the program, which involve identifying a real-world diagnostic 
shortfall, creating a novel solution to fill the need as well as conducting
clinical trials all within the standard 3 year time frame necessary to 
complete the degree. There is simply no other place on earth where research of
this type can progress from idea to human trials within this relatively short
time frame. While it is important to emphasize that this is exciting research,
it is also important to highlight where USAFA Cadets fit in.
 
OxCAHT follows a research model based on the mission of Engineering World
Health ``to inspire and mobilize the biomedical engineering community to improve
the quality of health care in resource-poor communities of the developing
world" (www.oxcaht.org) . The ``mobilization" aspect of the above mission
statement effectively created a desire within the Biomedical Engineering 
community to take the handcuffs off traditional academic research. This has
been accomplished by embracing an open-source paradigm for conducting research.
The result of this paradigm is the creation of a ``Wikipedia" for biomedical
engineering research called PhysioNet, which is an open-source repository of
academic endeavors to which anyone may contribute. This repository is
predominately comprised of computer source code and raw diagnostic data but
also contains hardware schematics, mechanical drawings and other documentation
of studies, technologies, and data contributing to the field of biomedical
engineering. The open-source nature of this research is particularly
interesting when incorporated with the research climate of USAFA.
 
The director of OxCAHT is in a unique situation. He is in a position to make
immense contributions to the wellbeing of entire populations, but he needs
help. Particularly, he needs help designing and implementing embedded solutions
to a plethora of diagnostic shortfalls. His center does an amazing job
identifying shortfalls and creating advanced algorithms that use
state-of-the-art machine learning and brilliant signal processing techniques to
properly analyze diagnostics data, but what he is missing is a team of
undergraduates willing to implement the OxCAHT solutions in hardware so that
they can be tested in the field. This is where USAFA Cadets come in. Once a
partnership is established between Oxford and USAFA, cadets could expect to
create hardware platforms that make OxCAHT's advanced processing techniques
work in real life. These devices could then be transferred back to Oxford,
where they will be administered to real patients by Oxford researchers.

Developing a research partnership between Oxford and USAFA based on open-source
technology is a relatively simple process to navigate, in fact I already have.
I have garnered approvals through Col Kraus, USAFA Office of Research, and the
USAFA Judge Advocate for the research partnership I have proposed (the
approvals can be found in Attachments 2 and 3
accordingly). Oxford-level research opportunities can
exist at USAFA, not as an arcane future promise, but as a detailed plan that
begins with choosing me for the Dean's Oxford Scholarship. Not only am I the
best candidate to develop and pursue these exciting research opportunities, I
am also the best candidate to bring these opportunities back to USAFA and
inspire a new generation of cadets dedicated to making tangible contributions
to the health and wellbeing of their fellow man.
 
I want to state unequivocally that my assignment at USAFA has easily been the
most rewarding experience of my career. I am truly passionate about the USAFA
Mission and I believe there is no greater professional fulfillment than to see
the sheer magnitude of positive influence and inspiration an officer can have
in a cadet's life. I've found that this is accomplished through tireless
devotion in and out of the classroom. My first tour has taught me to appreciate
the immense responsibility associated with shaping and motivating America's
future leaders, and it would be a privilege and an honor to be trusted with
that responsibility again by selecting me to bring Oxford-caliber opportunities
back to USAFA as a senior military faculty member.
 
My resume clearly illustrates that I have thrived during my time in DFEC as I
have won every award the department has to give as well as garnered national
level recognition as the Great Minds in Stem Most Promising Engineer of 2013,
but there is no formal record of the 3 accolades I hold most dear: I have been
afforded the opportunity to participate in the culmination of the USAFA Mission
through administering the Oath of Office thereby commissioning 3 cadets into
Active Duty. When considering who should administer the Oath, cadets are
instructed to select the officer whose inspiration had the greatest impact on
their development and whose leadership they would most like to emulate. It is
humbling to consider that these (now) Lieutenants in the classes of 2011, 2012
and 2013 selected me to be a part of the zenith of their USAFA careers and the
arbiter that bestows upon them the prize for which they have worked so hard.
Ultimately my personal commitment to the USAFA Mission has been recognized by
my fellow faculty and endorsed by cadets, whose interests USAFA exists to
serve. Choose me for the Dean's Oxford Scholarship and the real winners are the
cadets.

Gus Jones!

\newpage
\section*{Attachment 1 - Approved Oxford Research Proposal}
\label{sec:prop}
\centering
\includegraphics[scale=.85,clip=true,trim=1in .5in 1cm 0.4in]{MFR_ProposalforResearch_SilvaSIGNED.pdf}

\newpage
\section*{Attachment 2 - DF Research Office Approval to Conduct Research}
\label{sec:DFER}
\centering
\vspace{-1cm}
\hspace*{-1.5cm}\includegraphics[scale=.85]{DFER_email.pdf}
\includepdf[pages=2-3,scale=0.85,offset=-0 -25,pagecommand={}]{DFER_email.pdf}
\newpage
\section*{Attachment 3 - USAFA Judge Advocate Approval to Conduct Research}
\label{sec:JA}
\centering
\vspace{-1cm}
\hspace*{-1.5cm}\includegraphics[scale=.85]{email.pdf}
\includepdf[pages=2,scale=0.85,offset=-0 -25,pagecommand={}]{email.pdf}
%\label{sec:JA}

%\includepdf[scale=0.8,offset=-25 -25,pagecommand={\section{Attachment - Approved Oxford Research Proposal}\label{sec:prop}}]{MFR_ProposalforResearch_SilvaSIGNED.pdf}

\end{document}
